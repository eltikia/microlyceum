\documentclass[12pt]{article}
\usepackage{amsmath}
\usepackage{amssymb}
\usepackage{geometry}
\geometry{a4paper, margin=1in}

\begin{document}

\section*{Inequality Logic}
Real numbers \(x\) and \(y\) satisfy the inequalities:
\[
x^2 + y^2 \leq 50 \quad \text{and} \quad 2x + y \geq 15.
\]
Determine whether the following statements necessarily follow from these conditions:
\begin{enumerate}
    \item[(a)] \(x \leq 5\),
    \item[(b)] \(x^2 + y^2 \geq 45\),
    \item[(c)] \(x + y \leq 10\),
    \item[(d)] \(y \leq 5\).
\end{enumerate}

\section*{Solution}

\subsection*{Step 1: Analyzing the inequalities}
\begin{enumerate}
    \item \(x^2 + y^2 \leq 50\): This inequality defines a disk (or circle) of radius \(\sqrt{50}\) centered at the origin \((0, 0)\).
    \item \(2x + y \geq 15\): This inequality represents a half-plane. The boundary line \(2x + y = 15\) has a slope of \(-2\) and intercept \(y = 15\) when \(x = 0\). The solution set is the intersection of the disk \(x^2 + y^2 \leq 50\) and the half-plane \(2x + y \geq 15\).
\end{enumerate}

\subsection*{Step 2: Checking each statement}

\paragraph{(a) \(x \leq 5\):}

The largest possible value of \(x\) occurs along the line \(2x + y = 15\) where it intersects the boundary of the circle \(x^2 + y^2 = 50\). Substituting \(y = 15 - 2x\) into the circle equation:
\[
x^2 + (15 - 2x)^2 = 50.
\]
Expanding:
\[
x^2 + 225 - 60x + 4x^2 = 50 \implies 5x^2 - 60x + 225 = 50.
\]
Simplify:
\[
5x^2 - 60x + 175 = 0 \implies x^2 - 12x + 35 = 0.
\]
Factoring:
\[
(x - 7)(x - 5) = 0.
\]
The solutions are \(x = 7\) and \(x = 5\). To determine if \(x = 7\) is valid, calculate \(y = 15 - 2(7) = 1\). Substituting \(x = 7, y = 1\) into the circle equation:
\[
x^2 + y^2 = 7^2 + 1^2 = 49 + 1 = 50,
\]
which satisfies both inequalities. Hence, \(x = 7\) is valid, and \(x \leq 5\) does not necessarily follow. \textbf{Answer: False.}

\paragraph{(b) \(x^2 + y^2 \geq 45\):}

The minimum value of \(x^2 + y^2\) occurs at the intersection of \(2x + y = 15\) and \(x^2 + y^2 = 50\). From the solution to (a), all valid points lie on the circle boundary, where \(x^2 + y^2 = 50\), which is always greater than or equal to \(45\). \textbf{Answer: True.}

\paragraph{(c) \(x + y \leq 10\):}

The maximum value of \(x + y\) occurs at the boundary points. From (a), the valid points are:
\begin{itemize}
    \item At \(x = 5, y = 5\): \(x + y = 5 + 5 = 10\),
    \item At \(x = 7, y = 1\): \(x + y = 7 + 1 = 8\).
\end{itemize}
The maximum \(x + y = 10\), so \(x + y \leq 10\) holds for all points in the region. \textbf{Answer: True.}

\paragraph{(d) \(y \leq 5\):}

The maximum value of \(y\) occurs along the line \(2x + y = 15\). At \(x = 5\), \(y = 15 - 2(5) = 5\). Since no point in the solution region has \(y > 5\), the statement \(y \leq 5\) is valid. \textbf{Answer: True.}

\end{document}
