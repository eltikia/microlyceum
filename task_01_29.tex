\documentclass[12pt]{article}
\usepackage[utf8]{inputenc}
\usepackage[T1]{fontenc}
\usepackage{amsmath, amssymb}
\usepackage{geometry}
\usepackage{enumitem}
\usepackage{lmodern}
\geometry{a4paper, margin=1in}

\title{Analytical Comparison of Two Polynomial Sequences}
\author{}
\date{}

\begin{document}

\maketitle

\section{Problem Statement}

Let \( f_n = a n^2 + b n + 5 \) and \( g_n = c n^2 + d n + 1000 \), where \( a, b, c, d \in \mathbb{N}_0 \) (the set of non-negative integers). We aim to determine which of the following statements are logically equivalent to the assertion:

\begin{quote}
There exists \( n_0 \in \mathbb{N} \) such that for every \( n > n_0 \), we have \( g_n > f_n \).
\end{quote}

The candidate statements are:

\begin{enumerate}[label=\textbf{\alph*)}]
    \item One of the following two substatements holds:
    \begin{enumerate}[label=(\roman*)]
        \item \( a < c \)
        \item \( a = c \) and it is not true that \( b > d \)
    \end{enumerate}
    
    \item \( a \leq c \) and \( b \leq d \)
    
    \item There are infinitely many natural numbers \( n \in \mathbb{N} \) such that \( g_n > f_n \)
    
    \item The set \( \{ n \in \mathbb{N} : g_n \leq f_n \} \) has a maximum element
\end{enumerate}

\section{Analytical Setup}

Define the difference of the sequences:

\[
h(n) := g_n - f_n = (c - a)n^2 + (d - b)n + 995
\]

We analyze the sign of \( h(n) \) as \( n \to \infty \). The long-term behavior of \( h(n) \) is determined by its leading coefficient \( (c - a) \). We consider three cases:

\begin{itemize}
    \item If \( c > a \), then \( h(n) \to \infty \) as \( n \to \infty \), hence \( g_n > f_n \) for all sufficiently large \( n \).
    
    \item If \( c = a \), then \( h(n) = (d - b)n + 995 \) becomes a linear function:
    \begin{itemize}
        \item If \( d > b \), then \( h(n) \to \infty \), so \( g_n > f_n \) eventually.
        \item If \( d = b \), then \( h(n) = 995 > 0 \) for all \( n \).
        \item If \( d < b \), then \( h(n) \to -\infty \), so \( g_n < f_n \) for large \( n \).
    \end{itemize}
    
    \item If \( c < a \), then \( h(n) \to -\infty \), thus \( g_n < f_n \) for large \( n \).
\end{itemize}

Hence, the condition for \( g_n > f_n \) for all \( n > n_0 \) is:

\[
\boxed{c > a \quad \text{or} \quad (c = a \text{ and } d \geq b)}
\]

\section{Statement Analysis}

\subsection{Statement a)}

This is logically equivalent to:

\[
a < c \quad \text{or} \quad (a = c \text{ and } b \leq d)
\]

This exactly matches the necessary and sufficient condition derived above.

\textbf{Answer: Yes.}

\subsection{Statement b)}

This requires:

\[
a \leq c \quad \text{and} \quad b \leq d
\]

This condition is stronger than needed. For instance, if \( a < c \) but \( b > d \), the asymptotic inequality still holds. Therefore, this condition is not equivalent.

\textbf{Answer: No.}

\subsection{Statement c)}

This asserts that \( g_n > f_n \) holds for infinitely many natural numbers \( n \).

In the context of integer domains and polynomial functions with integer coefficients, if a quadratic function \( h(n) \) is positive for infinitely many natural numbers \( n \), and since \( h(n) \) can only change sign finitely many times, it must be eventually always positive. That is, the set \( \{ n \in \mathbb{N} : h(n) \leq 0 \} \) is finite.

Therefore, this condition \emph{is} equivalent to the original statement.

\textbf{Answer: Yes.}

\subsection{Statement d)}

This states that the set \( \{ n \in \mathbb{N} : g_n \leq f_n \} \) has a maximum element, i.e., is finite with an upper bound. This is equivalent to \( g_n > f_n \) for all sufficiently large \( n \).

\textbf{Answer: Yes.}

\end{document}
