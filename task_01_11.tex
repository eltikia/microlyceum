\documentclass{article}
\usepackage{amsmath}

\begin{document}

\textbf{System of equations}

For any real numbers \( x, y, z \) satisfying the system of equations:
\[
\begin{aligned}
1. & \quad x + y + z = 3, \\
2. & \quad 2x + 3y + 4z = 9, \\
3. & \quad 3x + y - z = 3,
\end{aligned}
\]
does the following hold true?
\begin{itemize}
    \item a) \( x = z \) (Yes/No)
    \item b) \( y = 1 \) (Yes/No)
    \item c) \( x = 1 \) (Yes/No)
    \item d) \( y = z \) (Yes/No)
\end{itemize}

\textbf{Step 1: Solve the system of equations}

First, we express \( z \) in terms of \( x \) and \( y \) using the first equation:
\[
z = 3 - x - y
\]

\textbf{Step 2: Substitute \( z \) into the other equations}

Substituting \( z \) into the second equation:
\[
2x + 3y + 4(3 - x - y) = 9
\]
Expanding:
\[
2x + 3y + 12 - 4x - 4y = 9
\]
Combining like terms:
\[
-2x - y + 12 = 9
\]
Rearranging:
\[
-2x - y = -3 \quad \Rightarrow \quad 2x + y = 3 \quad \text{(Equation 4)}
\]

Substituting \( z \) into the third equation:
\[
3x + y - (3 - x - y) = 3
\]
Expanding:
\[
3x + y - 3 + x + y = 3
\]
Combining like terms:
\[
4x + 2y - 3 = 3
\]
Rearranging:
\[
4x + 2y = 6 \quad \Rightarrow \quad 2x + y = 3 \quad \text{(Equation 5)}
\]

\textbf{Step 3: Analyze the relationships}

Both Equation 4 and Equation 5 are identical:
\[
2x + y = 3
\]
Thus, we can express \( y \) in terms of \( x \):
\[
y = 3 - 2x
\]

\textbf{Step 4: Substitute \( y \) back into the equation for \( z \)}

Substituting \( y \) back into the expression for \( z \):
\[
z = 3 - x - (3 - 2x) = 3 - x - 3 + 2x = x
\]

\textbf{Step 5: Summary of relationships}

From the derived equations, we have:
\[
y = 3 - 2x \quad \text{and} \quad z = x
\]

\textbf{Step 6: Evaluate the given options}

\begin{itemize}
    \item a) \( x = z \): \textbf{Yes} (since \( z = x \)).
    
    \item b) \( y = 1 \): To check:
    \[
    3 - 2x = 1 \quad \Rightarrow \quad 2x = 2 \quad \Rightarrow \quad x = 1 \quad \Rightarrow \quad y = 1 \text{ (This is true when } x = 1\text{)}
    \]
    However, \( y \) is not necessarily always equal to 1 for all values of \( x \). Thus, the statement is \textbf{No}.
    
    \item c) \( x = 1 \): This is not always true; \( x \) can take various values. Thus, the statement is \textbf{No}.
    
    \item d) \( y = z \): To check:
    \[
    3 - 2x = x \quad \Rightarrow \quad 3 = 3x \quad \Rightarrow \quad x = 1 \quad \Rightarrow \quad y = 1 \text{ and } z = 1 \text{ (This is true when } x = 1\text{)}
    \]
    However, this is not true for all \( x \). Thus, the statement is \textbf{No}.
\end{itemize}

\end{document}
