\documentclass{article}
\usepackage[utf8]{inputenc}
\usepackage{amsmath}
\usepackage{amsfonts}

\begin{document}

\section{Counting Real Solutions of a Quartic Equation}

For a given real number $a$, let $R(a)$ denote the number of real numbers $x$ such that $x^4 - 2x^2 = a$. Then:
\begin{itemize}
\item[a)] $R(-1) = 2$;
\item[b)] $R(0) = 3$;
\item[c)] $R(1) = 4$;
\item[d)] $R(-2) = 1$.
\end{itemize}

\section{Solution}

To find $R(a)$, we transform the equation into a function in the variable $y = x^2$. We have:

\[
x^4 - 2x^2 = y^2 - 2y = a \implies y^2 - 2y - a = 0
\]

Solving this quadratic equation, we use the quadratic formula:

\[
y = \frac{2 \pm \sqrt{(2)^2 - 4 \cdot 1 \cdot (-a)}}{2 \cdot 1} = 1 \pm \sqrt{1 + a}
\]

The real numbers $x$ corresponding to a given $y$ are such that $y = x^2 \geq 0$. Thus, we need to investigate the conditions for $y$:

\[
1 + a \geq 0 \implies a \geq -1
\]

Now we analyse specific values of $a$:

1. For $a = -1$:
\[
y = 1 \pm \sqrt{0} = 1
\]
There are two solutions: $x^2 = 1 \implies x = 1$ or $x = -1$.
\[
R(-1) = 2 \quad \text{(correct)}
\]

2. For $a = 0$:
\[
y = 1 \pm \sqrt{1} \implies y = 2 \text{ or } 0
\]
For $y = 2 \implies x^2 = 2 \implies x = \sqrt{2}, -\sqrt{2}$.
For $y = 0 \implies x = 0$.
\[
R(0) = 3 \quad \text{(correct)}
\]

3. For $a = 1$:
\[
y = 1 \pm \sqrt{2}
\]
For $y = 1 + \sqrt{2} \implies x^2 = 1 + \sqrt{2} \implies x = \sqrt{1+\sqrt{2}}, -\sqrt{1+\sqrt{2}}$.
For $y = 1 - \sqrt{2} \text{ (no solutions)}$.
\[
R(1) = 2 \quad \text{(incorrect)}
\]

4. For $a = -2$:
\[
y = 1 \pm \sqrt{-1} \text{ (no solutions)}
\]
\[
R(-2) = 0 \quad \text{(incorrect)}
\]

\end{document}