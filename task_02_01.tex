\documentclass[12pt]{article}
\usepackage{amsmath}
\usepackage{amssymb}
\usepackage[utf8]{inputenc}
\usepackage[margin=2.5cm]{geometry}

\title{Analytical Investigation of Integer Squares}
\author{}
\date{}

\begin{document}

\maketitle

\section*{Problem Statement}

Determine whether there exists an integer \( n \) such that:

\begin{enumerate}
    \item[(a)] The remainder of \( n^2 \) divided by 4 is equal to 3.
    \item[(b)] The sum of the digits of \( n^2 \) is equal to 36.
    \item[(c)] The sum of the digits of \( n^2 \) is equal to 21.
    \item[(d)] The remainder of \( n^2 \) divided by 8 is equal to 5.
\end{enumerate}

\section{Analysis}

\subsection{Question (a)}

\textbf{Is there an integer \( n \) such that \( n^2 \equiv 3 \pmod{4} \)?}

Let us analyze the possible values of \( n^2 \mod 4 \). For any integer \( n \), it is either even or odd.

\[
\begin{aligned}
n &\equiv 0 \pmod{2} \Rightarrow n^2 \equiv 0 \pmod{4}, \\
n &\equiv 1 \pmod{2} \Rightarrow n^2 \equiv 1 \pmod{4}.
\end{aligned}
\]

Therefore, the only possible values for \( n^2 \mod 4 \) are 0 and 1. Hence:

\textbf{Answer: No.}

\textbf{Explanation:} A square of an integer cannot be congruent to 3 modulo 4.

\subsection{Question (b)}

\textbf{Is there an integer \( n \) such that the sum of the digits of \( n^2 \) is 36?}

When n = 264, then \( n^2 \) = 69696. The sum of digits is 6 + 9 + 6 + 9 + 6 = 36; therefore, such a number exists.

Hence:

\textbf{Answer: Yes.}

\subsection{Question (c)}

\textbf{Is there an integer \( n \) such that the sum of the digits of \( n^2 \) is equal to 21?}

\[
S(n^2) \equiv n^2 \pmod{9}.
\]

If \( S(n^2) = 21 \), then \( n^2 \equiv 21 \equiv 3 \pmod{9} \).

We now examine whether 3 can be a quadratic residue modulo 9. The quadratic residues modulo 9 are obtained by squaring all residues modulo 9:
\[
\begin{aligned}
0^2 &\equiv 0, \\
1^2 &\equiv 1, \\
2^2 &\equiv 4, \\
3^2 &\equiv 0, \\
4^2 &\equiv 7, \\
5^2 &\equiv 7, \\
6^2 &\equiv 0, \\
7^2 &\equiv 4, \\
8^2 &\equiv 1.
\end{aligned}
\]

So the set of quadratic residues modulo 9 is:
\[
\{0, 1, 4, 7\}.
\]

Since \( 3 \not\in \{0, 1, 4, 7\} \), it follows that \( n^2 \not\equiv 3 \pmod{9} \).

\textbf{Answer: No.}

\textbf{Explanation:} Since the sum of the digits of \( n^2 \) equals 21, it must be congruent to 3 modulo 9. But 3 is not a quadratic residue modulo 9, so such an \( n \) does not exist.

\subsection{Question (d)}

\textbf{Is there an integer \( n \) such that \( n^2 \equiv 5 \pmod{8} \)?}

We examine the possible quadratic residues modulo 8:

\[
\begin{aligned}
n &\equiv 0 \pmod{8} \Rightarrow n^2 \equiv 0 \pmod{8}, \\
n &\equiv 1 \pmod{8} \Rightarrow n^2 \equiv 1 \pmod{8}, \\
n &\equiv 2 \pmod{8} \Rightarrow n^2 \equiv 4 \pmod{8}, \\
n &\equiv 3 \pmod{8} \Rightarrow n^2 \equiv 1 \pmod{8}, \\
n &\equiv 4 \pmod{8} \Rightarrow n^2 \equiv 0 \pmod{8}, \\
n &\equiv 5 \pmod{8} \Rightarrow n^2 \equiv 1 \pmod{8}, \\
n &\equiv 6 \pmod{8} \Rightarrow n^2 \equiv 4 \pmod{8}, \\
n &\equiv 7 \pmod{8} \Rightarrow n^2 \equiv 1 \pmod{8}.
\end{aligned}
\]

Thus, possible values of \( n^2 \mod 8 \) are:
\[
\{0, 1, 4\}.
\]

\textbf{Answer: No.}

\textbf{Explanation:} Since 5 is not a quadratic residue modulo 8, there does not exist any integer \( n \) such that \( n^2 \equiv 5 \pmod{8} \).

\section{Conclusion}

\begin{itemize}
    \item[(a)] No.
    \item[(b)] Yes.
    \item[(c)] No.
    \item[(d)] No.
\end{itemize}

\end{document}
