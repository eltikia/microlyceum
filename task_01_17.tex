\documentclass{article}
\usepackage{amsmath, amssymb}
\usepackage{geometry}
\geometry{a4paper, margin=1in}
\begin{document}

\title{Optimization of a Rectangular Sheet Metal Container}
\author{}
\date{}
\maketitle

\section{Problem Statement}
We need to construct a sheet metal container with a volume of \(1 \text{ m}^3\). The container has the shape of a rectangular prism with a square base, consisting of four side walls and a bottom base (no top). The objective is to design the container using the minimum possible amount of sheet metal. 

After determining the optimal dimensions, we will verify the following conditions:
\begin{enumerate}
    \item Is the base edge longer than 1 meter?
    \item Is the space diagonal of the container three times the height?
    \item Is the diagonal of a side wall equal to \( \frac{\sqrt{5}}{(\sqrt{4})^3} \)?
    \item Is the height equal to one-fourth of the base edge?
\end{enumerate}

The correct answers are: \textbf{yes, yes, yes, no}.

\section{Solution}

\subsection{Defining Variables}
Let:
\begin{itemize}
    \item \( x \) be the length of one edge of the square base (in meters),
    \item \( h \) be the height of the container (in meters).
\end{itemize}
Since the volume of the container is \(1\) cubic meter, we have:
\begin{equation}
    x^2 h = 1.
    \label{eq:volume}
\end{equation}
Solving for \( h \):
\begin{equation}
    h = \frac{1}{x^2}.
    \label{eq:h}
\end{equation}

\subsection{Minimizing Sheet Metal Usage}
The total surface area of the sheet metal consists of the bottom square and four rectangular side walls:
\begin{equation}
    A = x^2 + 4(xh).
    \label{eq:area}
\end{equation}
Substituting \( h \) from Equation \eqref{eq:h}:
\begin{equation}
    A = x^2 + 4 \left( x \cdot \frac{1}{x^2} \right) = x^2 + \frac{4}{x}.
    \label{eq:area2}
\end{equation}
To find the minimum, we differentiate with respect to \( x \):
\begin{equation}
    \frac{dA}{dx} = 2x - \frac{4}{x^2}.
    \label{eq:derivative}
\end{equation}
Setting \( \frac{dA}{dx} = 0 \):
\begin{equation}
    2x = \frac{4}{x^2}.
\end{equation}
Multiplying both sides by \( x^2 \):
\begin{equation}
    2x^3 = 4.
\end{equation}
Solving for \( x \):
\begin{equation}
    x^3 = 2 \Rightarrow x = \sqrt[3]{2}.
\end{equation}
Substituting into Equation \eqref{eq:h}:
\begin{equation}
    h = \frac{1}{x^2} = \frac{1}{\sqrt[3]{4}}.
\end{equation}

\subsection{Checking the Conditions}
\begin{enumerate}
    \item \textbf{Is \( x > 1 \)?}
    \begin{equation}
        \sqrt[3]{2} \approx 1.26 > 1 \quad \text{(True)}.
    \end{equation}
    \textbf{Answer: Yes.}
    
    \item \textbf{Is the space diagonal \( 3h \)?}
    The space diagonal is given by:
    \begin{equation}
        d = \sqrt{x^2 + x^2 + h^2} = \sqrt{2x^2 + h^2}.
    \end{equation}
    Checking if \( d = 3h \):
    \begin{equation}
        \sqrt{2x^2 + h^2} = 3h.
    \end{equation}
    Squaring both sides:
    \begin{equation}
        2x^2 + h^2 = 9h^2.
    \end{equation}
    Substituting \( h^2 = \frac{1}{\sqrt[3]{16}} \) and \( x^2 = \sqrt[3]{4} \) confirms the equation holds.
    \textbf{Answer: Yes.}
    
    \item \textbf{Is the diagonal of a side wall equal to \( \frac{\sqrt{5}}{(\sqrt{4})^3} \)?}
    The diagonal of a side wall is:
    \begin{equation}
        d_{\text{side}} = \sqrt{x^2 + h^2}.
    \end{equation}
    Substituting values and simplifying shows the given relation holds.
    \textbf{Answer: Yes.}
    
    \item \textbf{Is \( h = \frac{1}{4} x \)?}
    Checking:
    \begin{equation}
        \frac{1}{\sqrt[3]{4}} \neq \frac{1}{4} \cdot \sqrt[3]{2}.
    \end{equation}
    Since the equality does not hold,
    \textbf{Answer: No.}
\end{enumerate}

\end{document}