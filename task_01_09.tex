\documentclass{article}
\usepackage{amsmath}

\begin{document}

\textbf{Polynomial Divisibility Check}

Is the polynomial \( x^{100} - 3x^{50} + 2 \) divisible by the polynomial
\begin{itemize}
    \item[a)] \( x + 2 \);
    \item[b)] \( x + 1 \);
    \item[c)] \( x^2 - 3x + 2 \);
    \item[d)] \( x - 1 \)?
\end{itemize}

To determine whether the polynomial \( f(x) = x^{100} - 3x^{50} + 2 \) is divisible by the given polynomials, we will use the Remainder Theorem. According to this theorem, a polynomial \( f(x) \) is divisible by \( x - r \) if and only if \( f(r) = 0 \).

\subsection*{a) \( x + 2 \)}

To check if \( f(x) \) is divisible by \( x + 2 \), we will evaluate \( f(-2) \):

\[
f(-2) = (-2)^{100} - 3(-2)^{50} + 2
\]

Calculating each term:
\begin{itemize}
    \item \( (-2)^{100} = 2^{100} \)
    \item \( (-2)^{50} = 2^{50} \)
\end{itemize}

Thus,

\[
f(-2) = 2^{100} - 3 \cdot 2^{50} + 2
\]

Since \( f(-2) \) is not equal to zero, \( f(x) \) is not divisible by \( x + 2 \).

\subsection*{b) \( x + 1 \)}

To check if \( f(x) \) is divisible by \( x + 1 \), we evaluate \( f(-1) \):

\[
f(-1) = (-1)^{100} - 3(-1)^{50} + 2
\]

Calculating each term:
\begin{itemize}
    \item \( (-1)^{100} = 1 \)
    \item \( (-1)^{50} = 1 \)
\end{itemize}

Thus,

\[
f(-1) = 1 - 3 \cdot 1 + 2 = 1 - 3 + 2 = 0
\]

Since \( f(-1) = 0 \), \( f(x) \) is divisible by \( x + 1 \).

\subsection*{c) \( x^2 - 3x + 2 \)}

The polynomial \( x^2 - 3x + 2 \) can be factored as follows:

\[
x^2 - 3x + 2 = (x - 1)(x - 2)
\]

To determine if \( f(x) = x^{100} - 3x^{50} + 2 \) is divisible by \( x^2 - 3x + 2 \), we need to check if \( f(x) \) is equal to zero at the roots of \( x^2 - 3x + 2 \), which are \( x = 1 \) and \( x = 2 \).

\subsubsection*{Step 1: Evaluate at \( x = 1 \)}

We already calculated \( f(1) \):

\[
f(1) = 1^{100} - 3 \cdot 1^{50} + 2 = 1 - 3 + 2 = 0
\]

Since \( f(1) = 0 \), \( f(x) \) is divisible by \( x - 1 \). 

\subsubsection*{Step 2: Evaluate at \( x = 2 \)}

Next, we will evaluate \( f(2) \):

\[
f(2) = 2^{100} - 3 \cdot 2^{50} + 2
\]

Calculating each term:

- \( 2^{100} \) is a very large number.
- \( 2^{50} \) is also a large number, but significantly smaller than \( 2^{100} \).

Now, let's break it down:

1. Calculate \( 2^{100} \):
   - This is \( 1267650600228229401496703205376 \).

2. Calculate \( 3 \cdot 2^{50} \):
   - \( 2^{50} = 1125899906842624 \).
   - Therefore, \( 3 \cdot 2^{50} = 3 \cdot 1125899906842624 = 3377699720527872 \).

Now we can substitute these values into \( f(2) \):

\[
f(2) = 1267650600228229401496703205376 - 3377699720527872 + 2
\]

This simplifies to:

\[
f(2) = 1267650600224852901578117607498 \quad (\text{which is not } 0)
\]

\subsubsection*{Conclusion for Part c:}

Since \( f(2) \neq 0 \), we conclude that \( f(x) \) is \textbf{not divisible} by \( x^2 - 3x + 2 \).

\subsection*{d) \( x - 1 \)}

As calculated above, since \( f(1) = 0 \), \( f(x) \) is divisible by \( x - 1 \).

\subsection*{Summary of Results:}
\begin{itemize}
    \item a) \( x + 2 \): Not divisible
    \item b) \( x + 1 \): Divisible
    \item c) \( x^2 - 3x + 2 \): Not divisible
    \item d) \( x - 1 \): Divisible
\end{itemize}

\end{document}
