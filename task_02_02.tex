\documentclass[12pt]{article}
\usepackage{amsmath, amssymb}
\usepackage[utf8]{inputenc}
\usepackage{geometry}
\geometry{a4paper, margin=2.5cm}

\title{Analysis of the Existence of Real Solutions for Given Polynomial Equations}
\author{}
\date{}

\begin{document}

\maketitle

\section{Problem Statement}

Determine whether the following polynomial equations possess at least one real solution.

\begin{enumerate}
    \item[(a)] \( x^6 + 5x^3 + 5 = 0 \)
    \item[(b)] \( x^6 + 5x^2 + 5 = 0 \)
    \item[(c)] \( x^5 + 2x^4 + 3x^3 + 4x^2 + 5x + 6 = 0 \)
    \item[(d)] \( x^2 + 127x - 11 = 0 \)
\end{enumerate}

Each subproblem is examined separately below.

\section{Analysis and Solution}

\subsection{Equation (a): \( x^6 + 5x^3 + 5 = 0 \)}

Let us denote \( y = x^3 \). The equation becomes:
\[
y^2 + 5y + 5 = 0
\]
Solving this quadratic equation:
\[
y = \frac{-5 \pm \sqrt{25 - 4 \cdot 1 \cdot 5}}{2} = \frac{-5 \pm \sqrt{5}}{2}
\]
Both roots are real and negative, as the discriminant \( \sqrt{5} \) is real and the numerator is negative.

Now, recall that \( y = x^3 \), so we need to check whether the equation \( x^3 = y \) has a real solution for a given negative \( y \). Since the cube root function is defined and continuous for all real numbers and preserves the sign of negative inputs, both values of \( y \) correspond to real values of \( x \). Therefore, the original equation has two real solutions.

\textbf{Answer: Yes.} \\
\textit{Explanation:} The equation reduces to a quadratic in \( x^3 \), which has two real roots. Each of these corresponds to a real cube root, hence two real solutions to the original equation.

\subsection{Equation (b): \( x^6 + 5x^2 + 5 = 0 \)}

Substitute \( y = x^2 \), which implies \( y \geq 0 \). The equation becomes:
\[
y^3 + 5y + 5 = 0
\]

We study the function \( f(y) = y^3 + 5y + 5 \) on the interval \( [0, \infty) \). Note that:
\[
f(0) = 5 > 0, \quad \lim_{y \to \infty} f(y) = \infty
\]
Moreover, \( f(y) \) is strictly increasing on \( [0, \infty) \) since its derivative \( f'(y) = 3y^2 + 5 > 0 \) for all \( y \in \mathbb{R} \). Thus, \( f(y) > 0 \) for all \( y \geq 0 \), and the equation \( f(y) = 0 \) has no solution for \( y \geq 0 \).

\textbf{Answer: No.} \\
\textit{Explanation:} The transformed equation in \( y = x^2 \) has no non-negative real solution, and since \( x^2 \geq 0 \), the original equation has no real solution.

\subsection{Equation (c): \( x^5 + 2x^4 + 3x^3 + 4x^2 + 5x + 6 = 0 \)}

Let us denote this function by:
\[
f(x) = x^5 + 2x^4 + 3x^3 + 4x^2 + 5x + 6
\]

We analyze the behavior of this function. Since all coefficients are positive, we consider:
\[
f(x) > 0 \text{ for all } x > 0
\]

Now, we evaluate \( f(x) \) at several negative values:
\[
f(-1) = (-1)^5 + 2(-1)^4 + 3(-1)^3 + 4(-1)^2 + 5(-1) + 6 = -1 + 2 - 3 + 4 - 5 + 6 = 3 > 0
\]
\[
f(-2) = (-32) + 2(16) - 3(8) + 4(4) -10 + 6 = -32 + 32 - 24 + 16 -10 + 6 = -12
\]

So:
\[
f(-1) > 0, \quad f(-2) < 0
\]

Since the function is continuous, and it changes sign in the interval \((-2, -1)\), by the Intermediate Value Theorem, there exists at least one real root in this interval.

\textbf{Answer: Yes.} \\
\textit{Explanation:} The function is continuous and changes sign between \( x = -2 \) and \( x = -1 \), thus admitting at least one real root.

\subsection{Equation (d): \( x^2 + 127x - 11 = 0 \)}

This is a standard quadratic equation. Its discriminant is:
\[
\Delta = 127^2 + 4 \cdot 1 \cdot 11 = 16129 + 44 = 16173
\]

Since \( \Delta > 0 \), the equation has two distinct real roots.

\textbf{Answer: Yes.} \\
\textit{Explanation:} The discriminant is positive, ensuring two distinct real roots.

\end{document}
