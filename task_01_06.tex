\documentclass{article}
\usepackage{amsmath}
\usepackage{amsfonts}
\usepackage{amssymb}

\begin{document}

\title{Parity of Expressions}
\author{}
\date{}
\maketitle

\section{Problem Statement}

For any positive integer \( n \), the following number is even:
\begin{enumerate}
\item[a)] \( 3^n + n + 1 \)
\item[b)] \( n^2 + n \)
\item[c)] \( 3^n + n \)
\item[d)] \( 3^n + 5^n \)
\end{enumerate}

\section{Solution}

\subsection{a) \( 3^n + n + 1 \)}

- For \( n = 1 \):
\[
3^1 + 1 + 1 = 3 + 1 + 1 = 5 \quad \text{(odd)}
\]
- For \( n = 2 \):
\[
3^2 + 2 + 1 = 9 + 2 + 1 = 12 \quad \text{(even)}
\]
- For \( n = 3 \):
\[
3^3 + 3 + 1 = 27 + 3 + 1 = 31 \quad \text{(odd)}
\]

The expression does not consistently return even results.

\subsection{b) \( n^2 + n \)}

We can factor this expression:
\[
n^2 + n = n(n + 1)
\]
Since \( n \) and \( n + 1 \) are consecutive integers, one of them is always even. Thus, \( n(n + 1) \) is always even.

\subsection{c) \( 3^n + n \)}

- For \( n = 1 \):
\[
3^1 + 1 = 3 + 1 = 4 \quad \text{(even)}
\]
- For \( n = 2 \):
\[
3^2 + 2 = 9 + 2 = 11 \quad \text{(odd)}
\]
- For \( n = 3 \):
\[
3^3 + 3 = 27 + 3 = 30 \quad \text{(even)}
\]

This expression does not consistently return even results.

\subsection{d) \( 3^n + 5^n \)}

Both \( 3^n \) and \( 5^n \) are odd for any positive integer \( n \):
- Odd + Odd = Even.

Thus, the expression \( 3^n + 5^n \) is always even.

\section*{Summary}

The only expression that is guaranteed to be even for any positive integer \( n \) is:

\textbf{b) \( n^2 + n \)} and \textbf{d) \( 3^n + 5^n \)}.

\end{document}