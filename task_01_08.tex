\documentclass{article}
\usepackage{amsmath}

\begin{document}

\title{Checking the greatest common divisor}
\author{}
\date{}
\maketitle

\section*{Task}

Is the greatest common divisor given correctly:
\begin{itemize}
\item a) \( \text{GCD}(42, 84) = 14 \);
\item b) \( \text{GCD}(10^9 + 5, 10^9 + 35) = 15 \);
\item c) \( \text{GCD}(30, 42) = 6 \);
\item d) \( \text{GCD}(10^7 + 14, 10^7 + 21) = 7 \)? \end{itemize}

\section*{Solution:}

We will evaluate the validity of each of the theorems concerning the greatest common divisor (GCD), also known as the greatest common factor (GCD).

\subsection*{a) \( \text{GCD}(42, 84) = 14 \)}

To find the GCD of 42 and 84, we will apply prime factorization:

\[
42 = 2 \times 3 \times 7
\]
\[
84 = 2^2 \times 3 \times 7
\]

The common factors are: \( 2^1 \), \( 3^1 \), and \( 7^1 \).

GCD is calculated as:
\[
\text{GCD}(42, 84) = 2^1 \times 3^1 \times 7^1 = 42.
\]

So the statement \( \text{GCD}(42, 84) = 14 \) is invalid.

\subsection*{b) \( \text{GCD}(10^9 + 5, 10^9 + 35) = 15 \)}

Let \( a = 10^9 + 5 \) and \( b = 10^9 + 35 \).

We can simplify:
\[
b - a = (10^9 + 35) - (10^9 + 5) = 30.
\]

So we calculate:
\[
\text{GCD}(a, b) = \text{GCD}(10^9 + 5, 30).
\]

Now let's find the GCD of \( 10^9 + 5 \) and 30. The prime factorization of 30 is:
\[
30 = 2 \times 3 \times 5.
\]

Let's calculate \( 10^9 + 5 \mod 30 \):
\[
10^9 \mod 30 \equiv 10 \quad (\text{since } 10 \equiv 10 \mod 30).
\]
So,
\[
10^9 + 5 \equiv 10 + 5 \equiv 15 \mod 30.
\]

We calculate the GCD:
\[
\text{GCD}(15, 30) = 15.
\]

So the statement \( \text{GCD}(10^9 + 5, 10^9 + 35) = 15 \) is correct.

\subsection*{c) \( \text{GCD}(30, 42) = 6 \)}

We will apply prime factorization:
\[
30 = 2 \times 3 \times 5
\]
\[
42 = 2 \times 3 \times 7
\]

The common factors are \( 2 \) and \( 3 \):
\[
\text{GCD}(30, 42) = 2^1 \times 3^1 = 6.
\]

Therefore, the statement \( \text{GCD}(30, 42) = 6 \) is correct.

\subsection*{d) \( \text{GCD}(10^7 + 14, 10^7 + 21) = 7 \)}

Let \( c = 10^7 + 14 \) and \( d = 10^7 + 21 \).

We can simplify:
\[
d - c = (10^7 + 21) - (10^7 + 14) = 7.
\]

Now we calculate:
\[
\text{GCD}(c, d) = \text{GCD}(10^7 + 14, 7).
\]

First we calculate \( 10^7 + 14 \mod 7 \):
\[
10^7 \mod 7 \equiv 3 \quad (\text{since } 10 \equiv 3 \mod 7).
\]
So,
\[
10^7 + 14 \equiv 3 + 0 \equiv 3 \mod 7.
\]

We calculate the GCD:
\[
\text{GCD}(3, 7) = 1.
\]

So the statement \( \text{GCD}(10^7 + 14, 10^7 + 21) = 7 \) is incorrect.

\end{document}