\documentclass{article}
\usepackage[utf8]{inputenc}
\usepackage{amsmath}
\usepackage{amsfonts}
\usepackage{amssymb}

\begin{document}

\title{When A Becomes N}
\author{}
\date{}
\maketitle

We have a set \( A \) that is a subset of the natural numbers \( \mathbb{N} = \{1, 2, 3, \ldots\} \) and contains \( 0 \in A \). We will analyze the following conditions to determine if they imply that \( A = \mathbb{N} \). For each question (a, b, c, d), we will respond with "Yes" or "No" and provide detailed explanations of the solutions.

\section*{Questions and Answers}

\subsection*{Question a}
For every \( m \in A \), if there exists a natural number less than \( m \) in the set \( A \), then \( m + 1 \) is also in \( A \).

\textbf{Answer: No}

\textbf{Explanation:} The condition states that if there is a natural number \( n < m \) which belongs to \( A \), then \( m + 1 \) must also belong to \( A \). However, this does not guarantee that all natural numbers will be included in \( A \).

For instance, consider the set \( A = \{0, 2, 3, 4, 5, \ldots\} \). In this case:
\begin{itemize}
    \item For \( m = 2 \), there exists \( n = 0 \in A \) (where \( n < m \)), which implies that \( m + 1 = 3 \) must be in \( A \), and it is.
    \item However, \( 1 \notin A \), which means that \( A \neq \mathbb{N} \) since \( 1 \in \mathbb{N} \).
\end{itemize}

Thus, this condition does not ensure that \( A \) contains all natural numbers.

\subsection*{Question b}
For every \( m \in \mathbb{N} \), if all natural numbers less than \( m \) are in \( A \), then \( m \) is also in \( A \).

\textbf{Answer: Yes}

\textbf{Explanation:} This statement resembles the principle of mathematical induction. It asserts that if all natural numbers less than \( m \) are included in \( A \), then \( m \) must also be in \( A \).

We can illustrate this with mathematical induction:

\begin{itemize}
    \item \textbf{Base Case:} For \( m = 1 \), since \( 0 \in A \), we conclude that \( 1 \in A \).
    \item \textbf{Inductive Step:} Assume the statement is true for all \( k < m \). That is, if all numbers less than \( m \) are in \( A \), then \( m \) must also be in \( A \).
\end{itemize}

By induction, this means that every natural number must be in \( A \), leading to the conclusion that \( A = \mathbb{N} \).

\subsection*{Question c}
For every \( k \in \mathbb{N} \), there exists \( m > k \) such that \( m \in A \) and for every \( n > 0 \), if \( n \in A \), then \( n - 1 \in A \).

\textbf{Answer: Yes}

\textbf{Explanation:} This condition states that for each natural number \( k \), there exists a number \( m > k \) such that \( m \in A \) and if any \( n \in A \), then \( n - 1 \in A \).

This condition implies that \( A \) contains an infinite number of elements. If \( m \) is included in \( A \) and the property that \( n - 1 \in A \) holds, it suggests that we can keep finding numbers in \( A \) by decreasing from \( m \).

Therefore, this condition can be interpreted as a strong form of downward closure, which implies that if there are numbers in \( A \) greater than any \( k \), we can deduce that eventually all smaller natural numbers will also be included in \( A \). Thus, this condition does lead to the conclusion that \( A = \mathbb{N} \).

\subsection*{Question d}
There exists \( n \in \mathbb{N} \) such that \( n \in A \) implies \( n + 1 \in A \).

\textbf{Answer: No}

\textbf{Explanation:} The statement suggests that there is at least one natural number \( n \in A \) such that if \( n \) is in \( A \), then \( n + 1 \) must also be in \( A \). However, this condition alone does not guarantee that all natural numbers are included in \( A \).

For example, consider the set \( A = \{0, 1, 3\} \). Here, we can take \( n = 1 \), which is in \( A \) and implies that \( 2 \) should also be in \( A \). However, \( 2 \notin A \), and therefore not all natural numbers are in \( A \). This demonstrates that the condition does not ensure that \( A = \mathbb{N} \).

\end{document}