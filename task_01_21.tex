\documentclass{article}
\usepackage{amsmath}
\usepackage{amsfonts}
\usepackage{amssymb}

\title{Monotonicity of the function $f(x) = x^3 - x$ on given intervals}
\author{}
\date{}

\begin{document}

\maketitle

The task is to determine whether the function \( f(x) = x^3 - x \) is monotonic on the following intervals:

\begin{itemize}
    \item (a) \( \left( -\frac{1}{3}, \frac{1}{3} \right) \)
    \item (b) \( \left( \frac{1}{2}, 3 \right) \)
    \item (c) \( (0, 1) \)
    \item (d) \( (-7, -1) \)
\end{itemize}

We will determine whether the function is monotonic (increasing or decreasing) on each of these intervals by analyzing its derivative.

\section{Step 1: Finding the derivative of the function}

To analyze monotonicity, we first compute the derivative of the function \( f(x) = x^3 - x \).

The derivative is given by:

\[
f'(x) = \frac{d}{dx}(x^3 - x) = 3x^2 - 1
\]

\section{Step 2: Finding the critical points}

Next, we find the critical points by solving the equation \( f'(x) = 0 \):

\[
3x^2 - 1 = 0
\]

\[
3x^2 = 1
\]

\[
x^2 = \frac{1}{3}
\]

\[
x = \pm \frac{1}{\sqrt{3}} \approx \pm 0.577
\]

Thus, the critical points are \( x = \frac{1}{\sqrt{3}} \) and \( x = -\frac{1}{\sqrt{3}} \).

\section{Step 3: Analyzing the monotonicity on each interval}

We now analyze the monotonicity on each of the given intervals by checking the sign of the derivative in the relevant regions.

\subsection{Interval (a) \( \left( -\frac{1}{3}, \frac{1}{3} \right) \)}

The critical points \( x = \pm \frac{1}{\sqrt{3}} \approx \pm 0.577 \) lie outside this interval. Therefore, we do not expect any sign change of the derivative within this interval. We will check the sign of \( f'(x) \) at a point within this interval, for example at \( x = 0 \):

\[
f'(0) = 3(0)^2 - 1 = -1
\]

Since \( f'(x) = -1 \) is negative on the entire interval, the function is decreasing on \( \left( -\frac{1}{3}, \frac{1}{3} \right) \).

Therefore, the function is monotonic (decreasing) on this interval, and the answer to part (a) is **yes**.

\subsection{Interval (b) \( \left( \frac{1}{2}, 3 \right) \)}

In this interval, we have a critical point at \( x = \frac{1}{\sqrt{3}} \approx 0.577 \), which lies within the interval. We will check the sign of \( f'(x) \) at two points, for example at \( x = 0.6 \) and \( x = 2 \):

\[
f'(0.6) = 3(0.6)^2 - 1 = 3 \times 0.36 - 1 = 1.08 - 1 = 0.08
\]

\[
f'(2) = 3(2)^2 - 1 = 12 - 1 = 11
\]

The derivative is positive at both \( x = 0.6 \) and \( x = 2 \), which suggests that the function is increasing in this range. However, notice that \( \frac{1}{2} < \frac{1}{\sqrt{3}} \), and the derivative changes its sign at \( x = \frac{1}{\sqrt{3}} \), so we check the derivative at \( x = 0.3 \):

\[
f'(0.3) = 3(0.3)^2 - 1 = 3 \times 0.09 - 1 = 0.27 - 1 = -0.73
\]

Since the derivative is negative at \( x = 0.3 \) and positive at \( x = 0.6 \), the function changes its monotonicity on this interval. Therefore, the function is not monotonic on \( \left( \frac{1}{2}, 3 \right) \).

Thus, the answer to part (b) is **no**.

\subsection{Interval (c) \( (0, 1) \)}

The critical point \( x = \frac{1}{\sqrt{3}} \approx 0.577 \) lies within this interval. To check the monotonicity, we examine the derivative at two points: \( x = 0.3 \) and \( x = 0.8 \):

\[
f'(0.3) = 3(0.3)^2 - 1 = 3 \times 0.09 - 1 = 0.27 - 1 = -0.73
\]

\[
f'(0.8) = 3(0.8)^2 - 1 = 3 \times 0.64 - 1 = 1.92 - 1 = 0.92
\]

The derivative changes sign from negative at \( x = 0.3 \) to positive at \( x = 0.8 \), meaning the function is not monotonic on this interval.

Thus, the answer to part (c) is **no**.

\subsection{Interval (d) \( (-7, -1) \)}

In this interval, the critical points \( x = \pm \frac{1}{\sqrt{3}} \approx \pm 0.577 \) lie outside the interval. Therefore, the derivative does not change sign within this range. We check the sign of \( f'(x) \) at \( x = -2 \):

\[
f'(-2) = 3(-2)^2 - 1 = 12 - 1 = 11
\]

Since \( f'(x) = 11 \) is positive on the entire interval, the function is increasing on \( (-7, -1) \).

Thus, the answer to part (d) is **yes**.

\end{document}
