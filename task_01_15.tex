\documentclass[12pt]{article}
\usepackage{amsmath, amssymb}

\begin{document}

\section*{Existence of Minima}

Determine whether there exists a minimum value in the range (set of values) of the following functions defined on the set of real numbers:

\begin{enumerate}
    \item \( f(x) = 2^x + 2^{-x} \)
    \item \( f(x) = \frac{1}{x^2 + 1} \)
    \item \( f(x) = x^3 \)
    \item \( f(x) = x^2 \)
\end{enumerate}

\section*{Solutions}

\subsection*{a) \( f(x) = 2^x + 2^{-x} \)}

\textbf{Analysis}:

The function \( f(x) = 2^x + 2^{-x} \) is the sum of two exponential functions. To find its minimum value, we can use the properties of exponential functions.

\textbf{Finding the Minimum}:

To analyze the function, rewrite \( 2^{-x} \) as \( \frac{1}{2^x} \):
\[
f(x) = 2^x + \frac{1}{2^x}
\]
Let \( y = 2^x \). Since \( y > 0 \), the function can be expressed in terms of \( y \):
\[
f(y) = y + \frac{1}{y}
\]
To find the minimum value, take the derivative of \( f(y) \) with respect to \( y \) and set it to zero:
\[
f'(y) = 1 - \frac{1}{y^2}
\]
Setting \( f'(y) = 0 \):
\[
1 - \frac{1}{y^2} = 0 \implies y^2 = 1 \implies y = 1 \quad (y > 0)
\]
Now, evaluate \( f(y) \) at this critical point:
\[
f(1) = 1 + 1 = 2
\]

\textbf{Conclusion}: The minimum value of \( f(x) \) is 2, and therefore, there exists a minimum value in the range of the function.

\subsection*{b) \( f(x) = \frac{1}{x^2 + 1} \)}

\textbf{Analysis}:

The function \( f(x) = \frac{1}{x^2 + 1} \) is defined for all real numbers \( x \). The denominator \( x^2 + 1 \) is always positive and achieves its minimum value of 1 when \( x = 0 \).

\textbf{Finding the Minimum}:

The maximum value of \( f(x) \) occurs at \( x = 0 \):
\[
f(0) = \frac{1}{0^2 + 1} = 1
\]
As \( |x| \) increases, \( x^2 + 1 \) increases, leading to a decrease in \( f(x) \):
\[
\lim_{x \to \pm\infty} f(x) = \lim_{x \to \pm\infty} \frac{1}{x^2 + 1} = 0
\]
However, \( f(x) \) never actually reaches 0; it only approaches it.

\textbf{Conclusion}: Therefore, there is no minimum value (infimum) in the range of the function \( f(x) \) because while it can get arbitrarily close to 0, it never actually reaches it.

\subsection*{c) \( f(x) = x^3 \)}

\textbf{Analysis}:

The function \( f(x) = x^3 \) is a cubic polynomial. It is defined for all real \( x \).

\textbf{Finding the Minimum}:

As \( x \) approaches negative infinity, \( f(x) \) also approaches negative infinity:
\[
\lim_{x \to -\infty} f(x) = -\infty
\]
As \( x \) approaches positive infinity, \( f(x) \) approaches positive infinity:
\[
\lim_{x \to \infty} f(x) = \infty
\]
Since the function can take any real value, there is no minimum value in the range.

\textbf{Conclusion}: There is no minimum value in the range of the function \( f(x) \).

\subsection*{d) \( f(x) = x^2 \)}

\textbf{Analysis}:

The function \( f(x) = x^2 \) is a quadratic polynomial. It is defined for all real \( x \) and is always non-negative.

\textbf{Finding the Minimum}:

The minimum value occurs at \( x = 0 \):
\[
f(0) = 0^2 = 0
\]
As \( |x| \) increases, \( f(x) \) also increases:
\[
\lim_{x \to \pm\infty} f(x) = \infty
\]

\textbf{Conclusion}: The minimum value of \( f(x) \) is 0, and therefore, there exists a minimum value in the range of the function.

\section*{Summary of Results}

\begin{enumerate}
    \item \( f(x) = 2^x + 2^{-x} \): Exists minimum value, \( \mathbf{2} \)
    \item \( f(x) = \frac{1}{x^2 + 1} \): Does not exist minimum value
    \item \( f(x) = x^3 \): Does not exist minimum value
    \item \( f(x) = x^2 \): Exists minimum value, \( \mathbf{0} \)
\end{enumerate}

\end{document}
