\documentclass[a4paper,12pt]{article}
\usepackage{amsmath}
\usepackage{amsfonts}
\usepackage{amssymb}

\title{Problem: Intersection of Two Spheres}
\author{}
\date{}

\begin{document}

\maketitle

\section*{Problem:} 
Can the intersection of two spheres be
\begin{enumerate}
    \item a line?
    \item a plane?
    \item a circle?
    \item a single-point set?
\end{enumerate}

\section*{Solution:}

The intersection of two spheres in three-dimensional space can take on different forms, depending on the position and radius of the spheres. Here is an analysis of the mentioned cases:

\subsection*{a) Line:}
\begin{itemize}
    \item No. The intersection of two spheres cannot be a line. Spheres are three-dimensional objects, and their intersection, if not empty, forms at least a two-dimensional surface.
\end{itemize}

\subsection*{b) Plane:}
\begin{itemize}
    \item No. The intersection of two spheres also cannot be a plane. The intersection of spheres (if not empty) is at most a circle or a set of points, but not a plane.
\end{itemize}

\subsection*{c) Circle:}
\begin{itemize}
    \item Yes. The intersection of two spheres can be a circle. When two spheres intersect, their intersection forms a circle, which is a curved line in three-dimensional space.
\end{itemize}

\subsection*{d) Single-point set:}
\begin{itemize}
    \item Yes. The intersection of two spheres can be a single-point set. This happens when the spheres touch at exactly one point, which is possible when one sphere lies outside the other and they touch at precisely one point.
\end{itemize}

\section*{Summary:}
In summary, the intersection of two spheres can either be a circle or a single-point set, but it cannot be a line or a plane.

\end{document}
