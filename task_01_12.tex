\documentclass{article}
\usepackage{amsmath}

\title{Determining Rationality of Given Numbers}
\author{}
\date{}

\begin{document}

\maketitle

\section*{Problem}
Is the number rational for each of the following cases?

\begin{itemize}
    \item a) \( \sqrt{(\sqrt{3} - 2)^2 - \sqrt{3}} \)
    \item b) \( \sqrt{(1 - \sqrt{2})^2} + \sqrt{2} \)
    \item c) \( \sqrt{(2 - \sqrt{3})^2} + \sqrt{3} \)
    \item d) \( \sqrt{(\sqrt{2} - 1)^2} - \sqrt{2} \)
\end{itemize}

\section*{Solution}
To determine whether the given numbers are rational, we need to check if they can be simplified to a form that is a rational number (i.e., a number that can be expressed as a ratio of two integers).

\subsection*{a) \( \sqrt{(\sqrt{3} - 2)^2 - \sqrt{3}} \)}

1. First, calculate \( (\sqrt{3} - 2)^2 \):
   \[
   (\sqrt{3} - 2)^2 = 3 - 4\sqrt{3} + 4 = 7 - 4\sqrt{3}
   \]

2. Now subtract \( \sqrt{3} \):
   \[
   7 - 4\sqrt{3} - \sqrt{3} = 7 - 5\sqrt{3}
   \]

3. So we have:
   \[
   \sqrt{7 - 5\sqrt{3}}
   \]
   This number is irrational because \( 7 - 5\sqrt{3} \) is not a perfect square of a rational number.

\subsection*{b) \( \sqrt{(1 - \sqrt{2})^2} + \sqrt{2} \)}

1. Calculate \( (1 - \sqrt{2})^2 \):
   \[
   (1 - \sqrt{2})^2 = 1 - 2\sqrt{2} + 2 = 3 - 2\sqrt{2}
   \]

2. Note that \( \sqrt{(1 - \sqrt{2})^2} = |1 - \sqrt{2}| \). Since \( \sqrt{2} > 1 \), we have:
   \[
   |1 - \sqrt{2}| = \sqrt{2} - 1
   \]

3. Add \( \sqrt{2} \):
   \[
   (\sqrt{2} - 1) + \sqrt{2} = 2\sqrt{2} - 1
   \]
   This number is irrational because \( \sqrt{2} \) is irrational.

\subsection*{c) \( \sqrt{(2 - \sqrt{3})^2} + \sqrt{3} \)}

1. Calculate \( (2 - \sqrt{3})^2 \):
   \[
   (2 - \sqrt{3})^2 = 4 - 4\sqrt{3} + 3 = 7 - 4\sqrt{3}
   \]

2. We have:
   \[
   \sqrt{(2 - \sqrt{3})^2} = |2 - \sqrt{3}|
   \]
   Since \( 2 > \sqrt{3} \), we have:
   \[
   |2 - \sqrt{3}| = 2 - \sqrt{3}
   \]

3. Add \( \sqrt{3} \):
   \[
   (2 - \sqrt{3}) + \sqrt{3} = 2
   \]
   Since 2 is an integer, it is also a rational number.

\subsection*{d) \( \sqrt{(\sqrt{2} - 1)^2} - \sqrt{2} \)}

1. Calculate \( (\sqrt{2} - 1)^2 \):
   \[
   (\sqrt{2} - 1)^2 = 2 - 2\sqrt{2} + 1 = 3 - 2\sqrt{2}
   \]

2. We have:
   \[
   \sqrt{(\sqrt{2} - 1)^2} = |\sqrt{2} - 1|
   \]
   Since \( \sqrt{2} > 1 \), we have:
   \[
   |\sqrt{2} - 1| = \sqrt{2} - 1
   \]

3. Subtract \( \sqrt{2} \):
   \[
   (\sqrt{2} - 1) - \sqrt{2} = -1
   \]
   Since -1 is an integer, it is also a rational number.

\end{document}
