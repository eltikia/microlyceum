\documentclass[12pt]{article}
\usepackage[utf8]{inputenc}
\usepackage{amsmath}
\usepackage{geometry}
\usepackage{parskip}

\geometry{a4paper, margin=2.5cm}

\title{Analytical Solution to a Combinatorial Game Problem}
\author{}
\date{}

\begin{document}

\maketitle

\section{Problem Description}

Consider a deterministic, perfect-information game involving two players, Player A and Player B, who alternate turns. Player A initiates the game with $n$ candies.

On each move, the player in possession of the candies must eat either 2 or 3 candies and then pass the remainder to the opponent. The game terminates when only 1 candy remains, resulting in a draw. The player who consumes the last candy (either the second or third last in the penultimate turn) is declared the winner.

We are tasked with evaluating the truth of four given statements concerning different initial values of $n$. Each statement assumes that both players play optimally.

\section{Analytical Strategy}

Let us define the result of the game starting with $n$ candies under optimal play by both players. The outcomes are classified into three categories:

\begin{itemize}
    \item \textbf{A win for Player A} — Player A can force a win regardless of Player B’s strategy.
    \item \textbf{A win for Player B} — Player B can force a win regardless of Player A’s strategy.
    \item \textbf{A draw} — Both players play optimally, and the game ends with one remaining candy.
\end{itemize}

We begin by analyzing the base cases:
\begin{itemize}
    \item If $n = 1$, the game immediately ends in a draw.
    \item If $n = 2$ or $n = 3$, Player A can consume all candies and win.
\end{itemize}

For $n > 3$, Player A must choose either 2 or 3 candies to eat and leave the remaining $n-2$ or $n-3$ candies to Player B, who then follows the same rules. Thus, the outcome for a given $n$ depends recursively on the results for $n-2$ and $n-3$.

Through induction or by building a table of outcomes for consecutive values of $n$, we observe the following periodic pattern with modulus 5:

\begin{itemize}
    \item If $n \bmod 5 = 1$, the game ends in a draw.
    \item If $n \bmod 5 = 2$ or $n \bmod 5 = 3$, Player A wins.
    \item If $n \bmod 5 = 4$, the game ends in a draw.
    \item If $n \bmod 5 = 0$, Player B wins.
\end{itemize}

This pattern has been validated by exhaustively computing the outcomes for values of $n$ from 1 through 20 and observing that the results repeat every 5 numbers.

\section{Evaluation of Statements}

\subsection{Statement a}

\textit{If $n = 634$, then if both players play flawlessly, the game will end in a draw.}

\textbf{Calculation:}
\[
634 \bmod 5 = 4
\]

\textbf{Conclusion:} According to the established pattern, if $n \bmod 5 = 4$, then the game ends in a draw.

\textbf{Answer: Yes}

\subsection{Statement b}

\textit{If $n = 63$, then if both players play flawlessly, Player A will win.}

\textbf{Calculation:}
\[
63 \bmod 5 = 3
\]

\textbf{Conclusion:} If $n \bmod 5 = 3$, then Player A can force a win under optimal strategy.

\textbf{Answer: Yes}

\subsection{Statement c}

\textit{If $n = 364$, then if both players play flawlessly, Player A will win.}

\textbf{Calculation:}
\[
364 \bmod 5 = 4
\]

\textbf{Conclusion:} If $n \bmod 5 = 4$, the game ends in a draw.

\textbf{Answer: No}

\subsection{Statement d}

\textit{If $n = 630$, then if both players play flawlessly, Player B will win.}

\textbf{Calculation:}
\[
630 \bmod 5 = 0
\]

\textbf{Conclusion:} If $n \bmod 5 = 0$, Player B wins under optimal play.

\textbf{Answer: Yes}

\section{Final Remarks}

The key to solving this problem lies in recognizing the periodic nature of the outcome based on $n \bmod 5$. This insight permits a complete classification of all initial values of $n$ into one of the three possible outcomes. The periodicity results from the recursive structure of the decision tree rooted in the allowed actions (eating 2 or 3 candies) and the terminal condition (one candy remaining).

\end{document}
