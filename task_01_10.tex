\documentclass{article}
\usepackage{amsmath}

\title{Length of the Lateral Edge of a Regular Pyramid with an \( n \)-sided Base}
\author{}
\date{}

\begin{document}

\maketitle

Let \( k(n) \) be the length of the lateral edge of a regular pyramid with an \( n \)-sided base, in which both the height and the length of the base edge are equal to 1. We aim to determine the values of \( k(n) \) for \( n = 3, 4, 5, 6 \).

\section*{Step 1: Pyramid Structure and Geometry}

\subsection*{Characteristics of the Regular Pyramid}
\begin{itemize}
    \item The base is a regular \( n \)-gon.
    \item The height \( h \) of the pyramid is equal to 1.
    \item The length of each edge of the base polygon is also equal to 1.
\end{itemize}

\subsection*{Lateral Edge}
The lateral edge connects the apex \( O \) of the pyramid to a vertex \( A \) of the base.

\section*{Step 2: Identifying Key Elements}

\begin{itemize}
    \item Let \( O \) be the apex of the pyramid and \( G \) be the center of the base polygon.
    \item The length of the lateral edge \( k(n) \) can be calculated using the Pythagorean theorem in triangle \( OGA \).
\end{itemize}

\section*{Step 3: Calculating \( k(n) \)}

The length of the lateral edge \( k(n) \) can be expressed as:

\[
k(n) = \sqrt{h^2 + GA^2}
\]

where:
\begin{itemize}
    \item \( h = 1 \) (the height of the pyramid).
    \item \( GA \) is the distance from the center \( G \) of the base to a vertex \( A \).
\end{itemize}

\subsection*{Step 3.1: Finding \( GA \)}

To find \( GA \), we consider the geometry of the regular \( n \)-gon inscribed in a circle of radius \( R \):

\begin{itemize}
    \item The radius \( R \) of the circumcircle of the regular \( n \)-gon can be found using the formula:
    \[
    R = \frac{a}{2 \sin(\pi/n)}
    \]
    where \( a \) is the length of a side of the polygon. Since \( a = 1 \), we have:
    \[
    R = \frac{1}{2 \sin(\pi/n)}
    \]
    \item This distance \( GA \) is equal to \( R \).
\end{itemize}

\section*{Step 4: Evaluating \( k(n) \) for \( n = 3, 4, 5, 6 \)}

\subsection*{For \( n = 3 \) (Triangle)}
The circumradius \( R \) is:
\[
R = \frac{1}{2 \sin(\pi/3)} = \frac{1}{2 \cdot \frac{\sqrt{3}}{2}} = \frac{1}{\sqrt{3}}
\]
Thus, \( GA = R = \frac{1}{\sqrt{3}} \). Therefore, we compute \( k(3) \):
\[
k(3) = \sqrt{1^2 + \left(\frac{1}{\sqrt{3}}\right)^2} = \sqrt{1 + \frac{1}{3}} = \sqrt{\frac{4}{3}} = \frac{2}{\sqrt{3}}.
\]

\subsection*{For \( n = 4 \) (Square)}
The circumradius \( R \) is:
\[
R = \frac{1}{2 \sin(\pi/4)} = \frac{1}{2 \cdot \frac{\sqrt{2}}{2}} = \frac{1}{\sqrt{2}}.
\]
Thus, \( GA = R = \frac{1}{\sqrt{2}} \). Therefore, we compute \( k(4) \):
\[
k(4) = \sqrt{1^2 + \left(\frac{1}{\sqrt{2}}\right)^2} = \sqrt{1 + \frac{1}{2}} = \sqrt{\frac{3}{2}} = \frac{\sqrt{3}}{\sqrt{2}}.
\]

\subsection*{For \( n = 5 \) (Pentagon)}
The circumradius \( R \) is:
\[
R = \frac{1}{2 \sin(\pi/5)}.
\]
Since \( \sin(\pi/5) \approx 0.5878 \), we compute:
\[
R \approx \frac{1}{2 \cdot 0.5878} \approx \frac{1}{1.1756} \approx 0.849.
\]
Thus, \( GA \approx 0.849 \). Therefore, we compute \( k(5) \):
\[
k(5) = \sqrt{1 + (0.849)^2} \approx \sqrt{1 + 0.720} \approx \sqrt{1.720} \approx \sqrt{2}.
\]

\subsection*{For \( n = 6 \) (Hexagon)}
The circumradius \( R \) is:
\[
R = \frac{1}{2 \sin(\pi/6)} = \frac{1}{2 \cdot \frac{1}{2}} = 1.
\]
Thus, \( GA = 1 \). Therefore, we compute \( k(6) \):
\[
k(6) = \sqrt{1^2 + 1^2} = \sqrt{1 + 1} = \sqrt{2}.
\]

\section*{Summary of Results}

The correct values for \( k(n) \) are as follows:
\begin{itemize}
    \item \( k(3) = \frac{2}{\sqrt{3}} \) (Correct)
    \item \( k(4) = \frac{\sqrt{3}}{\sqrt{2}} \) (Correct)
    \item \( k(5) = \sqrt{2} \) (Incorrect as noted)
    \item \( k(6) = 1 \) (Incorrect, should be \( \sqrt{2} \))
\end{itemize}

Thus, the correct statements from the original problem are:
\begin{itemize}
    \item a) No
    \item b) No
    \item c) Yes
    \item d) Yes
\end{itemize}

\end{document}
