\documentclass[12pt]{article}
\usepackage[a4paper,margin=2.5cm]{geometry}
\usepackage{amsmath,amssymb}
\usepackage{mathrsfs}
\usepackage{enumerate}

\title{Analytical Problem Solving: Degree Constraints in an Undirected Graph}
\author{}
\date{}

\begin{document}

\maketitle

\section{Problem Statement}

At a certain party, there were 19 people. Each person knew at least one and at most three of the other attendees. It is known that if person $X$ knows person $Y$, then $Y$ also knows $X$. Does this imply that:

\begin{enumerate}[a)]
    \item There was an \textbf{odd number} of attendees who knew \textbf{only one} other person?
    \item There was an \textbf{even number} of attendees who knew \textbf{three} other people?
    \item The number of attendees who knew \textbf{only one} other person was the \textbf{same} as the number of attendees who knew \textbf{three} other people?
    \item There was an \textbf{odd number} of attendees who knew \textbf{exactly two} other people?
\end{enumerate}

\section{Reformulation Using Graph Theory}

We can model this situation as a simple undirected graph \( G = (V, E) \), where:
\begin{itemize}
    \item Each vertex \( v \in V \) represents a person at the party.
    \item An edge \( \{u, v\} \in E \) represents a mutual acquaintance between \( u \) and \( v \).
\end{itemize}

We are told that:
\begin{itemize}
    \item \( |V| = 19 \)
    \item The degree \( \deg(v) \in \{1,2,3\} \) for every vertex \( v \in V \)
\end{itemize}

Let us define:
\[
n_1 = \text{number of vertices with degree } 1, \\
n_2 = \text{number of vertices with degree } 2, \\
n_3 = \text{number of vertices with degree } 3.
\]
Since there are 19 people:
\[
n_1 + n_2 + n_3 = 19 \tag{1}
\]
The sum of all vertex degrees in a graph equals twice the number of edges:
\[
\sum_{v \in V} \deg(v) = 2|E| \Rightarrow n_1 + 2n_2 + 3n_3 = 2|E| \tag{2}
\]

We now analyze the individual questions using these relations.

\section{Analysis}

\subsection{Question a: Was there an odd number of attendees who knew only one other person?}

We want to determine whether \( n_1 \) must be odd.

Consider the sum of all degrees:
\[
S = n_1 + 2n_2 + 3n_3
\]
This sum must be even, since it equals twice the number of edges.

Let us consider the parity (even or odd) of this sum modulo 2:
\[
n_1 + 2n_2 + 3n_3 \equiv n_1 + 3n_3 \pmod{2}
\]
(since \( 2n_2 \equiv 0 \mod 2 \))

So for the total degree sum to be even:
\[
n_1 + 3n_3 \equiv 0 \mod 2
\Rightarrow n_1 + n_3 \equiv 0 \mod 2 \tag{3}
\]

This tells us that \( n_1 \equiv n_3 \mod 2 \). Therefore, \( n_1 \) and \( n_3 \) must have the same parity.

Since we do not yet know \( n_3 \), we cannot conclude definitively that \( n_1 \) is odd. For example:
\begin{itemize}
    \item If \( n_1 = 3 \), \( n_3 = 3 \): both odd.
    \item If \( n_1 = 2 \), \( n_3 = 2 \): both even.
\end{itemize}

Hence, different values of \( n_1 \) with different parity are possible under the given constraints. 

\textbf{Answer: No.}

\subsection{Question b: Was there an even number of attendees who knew three other people?}

This is equivalent to asking whether \( n_3 \) must be even.

From equation (3) again:
\[
n_1 + n_3 \equiv 0 \mod 2 \Rightarrow n_1 \equiv n_3 \mod 2
\]
This implies \( n_3 \equiv n_1 \mod 2 \), so again the parity of \( n_3 \) depends on that of \( n_1 \).

If \( n_1 \) is odd, then \( n_3 \) is odd.
If \( n_1 \) is even, then \( n_3 \) is even.

Thus, both even and odd values for \( n_3 \) are consistent with the constraints.

\textbf{Answer: No.}

\subsection{Question c: Were there as many attendees who knew only one other person as those who knew three?}

This asks whether \( n_1 = n_3 \) must hold.

Equation (3) tells us \( n_1 \equiv n_3 \mod 2 \), which only implies that their parity is the same, not that they are equal.

For example, the following configurations are valid:
\begin{itemize}
    \item \( n_1 = 2 \), \( n_2 = 15 \), \( n_3 = 2 \): sum of degrees is \( 2 + 30 + 6 = 38 \), even.
    \item \( n_1 = 4 \), \( n_2 = 11 \), \( n_3 = 4 \): total 19 nodes, degree sum \( 4 + 22 + 12 = 38 \), even.
\end{itemize}

In both cases, \( n_1 = n_3 \), but it is not required by the constraints.

Also:
\begin{itemize}
    \item \( n_1 = 2 \), \( n_2 = 14 \), \( n_3 = 3 \): total 19 nodes, degree sum \( 2 + 28 + 9 = 39 \), odd — invalid.
    \item \( n_1 = 2 \), \( n_2 = 13 \), \( n_3 = 4 \): total 19 nodes, degree sum \( 2 + 26 + 12 = 40 \), even — valid, but now \( n_1 \ne n_3 \).
\end{itemize}

Hence, equality is not enforced.

\textbf{Answer: No.}

\subsection{Question d: Was there an odd number of attendees who knew exactly two other people?}

We now want to determine whether \( n_2 \) must be odd.

We return to the degree sum:
\[
n_1 + 2n_2 + 3n_3 \equiv 0 \mod 2
\Rightarrow n_1 + 3n_3 \equiv 0 \mod 2
\Rightarrow n_1 + n_3 \equiv 0 \mod 2
\Rightarrow n_1 + n_3 \text{ even}
\]
Since total number of people is 19, we have:
\[
n_1 + n_2 + n_3 = 19 \Rightarrow n_2 = 19 - (n_1 + n_3)
\]
Then:
\[
n_2 \equiv 19 - (n_1 + n_3) \mod 2
\]
Because \( n_1 + n_3 \) is even, this gives:
\[
n_2 \equiv 19 - \text{even} \equiv 1 \mod 2
\Rightarrow n_2 \text{ is odd}
\]

Therefore, the number of attendees who knew exactly two others \textbf{must} be odd.

\textbf{Answer: Yes.}

\end{document}
