\documentclass{article}
\usepackage{amsmath}
\begin{document}

\section*{Trigonometry in Arithmetic Sequences}
Determine whether the following three numbers form an arithmetic sequence in the given order:

\begin{enumerate}
    \item $\cos^2 34^\circ, \cos^2 45^\circ, \cos^2 56^\circ$;
    \item $\sin 10^\circ, \sin 30^\circ, \sin 50^\circ$;
    \item $\tan 150^\circ, \tan 30^\circ, \tan 60^\circ$;
    \item $\cos 0^\circ, \cos 60^\circ, \cos 90^\circ$.
\end{enumerate}

\section*{Solution}
An arithmetic sequence is defined by the condition that the difference between consecutive terms remains constant. That is, for three numbers $a, b, c$, they form an arithmetic sequence if and only if:
\[
2b = a + c.
\]
We check this condition for each case.

\subsection*{(a) Checking $\cos^2 34^\circ, \cos^2 45^\circ, \cos^2 56^\circ$}
We verify whether:
\[
2 \cos^2 45^\circ = \cos^2 34^\circ + \cos^2 56^\circ.
\]
Since:
\[
\cos 45^\circ = \frac{\sqrt{2}}{2}, \quad \cos^2 45^\circ = \frac{1}{2},
\]
and using trigonometric identities,
\[
\cos 34^\circ = \sin 56^\circ, \quad \cos 56^\circ = \sin 34^\circ,
\]
we find that the equation holds. Thus, these numbers \textbf{form} an arithmetic sequence.

\subsection*{(b) Checking $\sin 10^\circ, \sin 30^\circ, \sin 50^\circ$}
We verify whether:
\[
2 \sin 30^\circ = \sin 10^\circ + \sin 50^\circ.
\]
Since:
\[
\sin 30^\circ = \frac{1}{2},
\]
and using the sum-to-product identity:
\[
\sin A + \sin B = 2 \sin \left( \frac{A+B}{2} \right) \cos \left( \frac{A-B}{2} \right),
\]
for $A = 50^\circ$ and $B = 10^\circ$,
\[
\sin 50^\circ + \sin 10^\circ = 2 \sin 30^\circ \cos 20^\circ = 2 \times \frac{1}{2} \times \cos 20^\circ = \cos 20^\circ.
\]
Since $\cos 20^\circ \neq 1$, the equality does not hold, meaning these numbers \textbf{do not form} an arithmetic sequence.

\subsection*{(c) Checking $\tan 150^\circ, \tan 30^\circ, \tan 60^\circ$}
We verify whether:
\[
2 \tan 30^\circ = \tan 150^\circ + \tan 60^\circ.
\]
Since:
\[
\tan 30^\circ = \frac{1}{\sqrt{3}}, \quad \tan 60^\circ = \sqrt{3}, \quad \tan 150^\circ = -\frac{1}{\sqrt{3}},
\]
we check:
\[
2 \times \frac{1}{\sqrt{3}} = -\frac{1}{\sqrt{3}} + \sqrt{3}.
\]
This simplifies to:
\[
\frac{2}{\sqrt{3}} = \sqrt{3} - \frac{1}{\sqrt{3}} = \frac{3}{\sqrt{3}} - \frac{1}{\sqrt{3}} = \frac{2}{\sqrt{3}},
\]
which is true. Thus, these numbers \textbf{form} an arithmetic sequence.

\subsection*{(d) Checking $\cos 0^\circ, \cos 60^\circ, \cos 90^\circ$}
We verify whether:
\[
2 \cos 60^\circ = \cos 0^\circ + \cos 90^\circ.
\]
Since:
\[
\cos 0^\circ = 1, \quad \cos 60^\circ = \frac{1}{2}, \quad \cos 90^\circ = 0,
\]
we check:
\[
2 \times \frac{1}{2} = 1 + 0.
\]
This simplifies to:
\[
1 = 1,
\]
which is true. Thus, these numbers \textbf{form} an arithmetic sequence.

\section*{Final Answer}
\begin{enumerate}
    \item Yes
    \item No
    \item Yes
    \item Yes
\end{enumerate}

\end{document}

