\documentclass{article}
\usepackage{amsmath}
\usepackage{amsfonts}
\usepackage{graphicx}

\title{Verification of Logarithmic Identities}
\author{}
\date{}

\begin{document}

\maketitle

\section{Problem Statement}

Determine the truth of the following equalities:
\begin{enumerate}
\item[a)] \( 3^n + n + 1 \)
\item[b)] \( n^2 + n \)
\item[c)] \( 3^n + n \)
\item[d)] \( 3^n + 5^n \)
\end{enumerate}

\section{Solution}

\subsection{a) $\log_{7} (\sqrt{5}) = \sqrt{\log_{7} (5)}$}

Using the property of logarithms, we have:
\[
\log_{7} (\sqrt{5}) = \log_{7} (5^{1/2}) = \frac{1}{2} \log_{7} (5)
\]
Comparing this with the right-hand side, we find:
\[
\sqrt{\log_{7} (5)} \text{ is not equal to } \frac{1}{2} \log_{7} (5) \text{ in general.}
\]
Thus, this equality is \textbf{false}.

\subsection{b) $\log_{7} (2) + \log_{7} (3) = \log_{7} (6)$}

Using the property of logarithms that states $\log_{b} (x) + \log_{b} (y) = \log_{b} (xy)$, we find:
\[
\log_{7} (2) + \log_{7} (3) = \log_{7} (2 \cdot 3) = \log_{7} (6)
\]
Thus, this equality is \textbf{true}.

\subsection{c) $\log_{2} (\log_{4} (16)) = \log_{4} (\log_{2} (16))$}

Calculating each side:
- For the left side:
\[
\log_{4} (16) = \log_{4} (4^2) = 2 \implies \log_{2} (2) = 1
\]

- For the right side:
\[
\log_{2} (16) = \log_{2} (2^4) = 4 \implies \log_{4} (4) = 1
\]

Thus, both sides equal 1, so this equality is \textbf{true}.

\subsection{d) $\log_{7} (2) \cdot \log_{7} (3) = \log_{7} (5)$}

Using the change of base formula and properties of logarithms:
\[
\log_{7} (2) \cdot \log_{7} (3) \text{ does not equal } \log_{7} (5) \text{ in general.}
\]
Thus, this equality is \textbf{false}.

\end{document}