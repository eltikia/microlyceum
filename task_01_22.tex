\documentclass{article}
\usepackage{amsmath}
\usepackage{amsfonts}
\usepackage{amssymb}

\begin{document}

\title{Analysis of the Convergence of Sequences}
\author{}
\date{}
\maketitle

\section*{Task and Solution}

We are tasked with answering whether the sequence \( (a_n) \) defined by the following formulas converges to a finite limit.

Answer with "yes" or "no" for each case, and provide detailed explanations of the solutions.

\begin{enumerate}
    \item \( a_n = \log_2(n) \)
    \item \( a_n = 2^{-n} \)
    \item \( a_n = \frac{2n^2 + 3n^3 + 4}{n^2 + 1} \)
    \item \( a_n = \log_{n+1}(3) \)
\end{enumerate}

For each sequence, we will rigorously analyze its behavior as \( n \to \infty \) using exact calculations and formulas.

\section*{Solution}

\subsection*{1. Sequence \( a_n = \log_2(n) \)}

We are given the sequence \( a_n = \log_2(n) \). The logarithmic function \( \log_2(n) \) increases without bound as \( n \) increases. Specifically:

\[
a_n = \log_2(n) = \frac{\ln(n)}{\ln(2)}
\]

As \( n \to \infty \), the natural logarithm \( \ln(n) \to \infty \), so:

\[
\lim_{n \to \infty} \log_2(n) = \infty
\]

Thus, the sequence \( a_n = \log_2(n) \) does not converge to a finite limit.

\textbf{Answer: No}

\subsection*{2. Sequence \( a_n = 2^{-n} \)}

Now, consider the sequence \( a_n = 2^{-n} \). This sequence is an exponential decay. We can express it as:

\[
a_n = 2^{-n} = \frac{1}{2^n}
\]

As \( n \to \infty \), the term \( 2^n \) grows exponentially, so:

\[
\lim_{n \to \infty} 2^{-n} = 0
\]

Therefore, the sequence \( a_n = 2^{-n} \) converges to 0.

\textbf{Answer: Yes}

\subsection*{3. Sequence \( a_n = \frac{2n^2 + 3n^3 + 4}{n^2 + 1} \)}

Next, we consider the sequence \( a_n = \frac{2n^2 + 3n^3 + 4}{n^2 + 1} \). We begin by dividing both the numerator and denominator by \( n^2 \):

\[
a_n = \frac{2n^2 + 3n^3 + 4}{n^2 + 1} = \frac{n^2(2 + 3n + \frac{4}{n^2})}{n^2(1 + \frac{1}{n^2})}
\]

This simplifies to:

\[
a_n = \frac{2 + 3n + \frac{4}{n^2}}{1 + \frac{1}{n^2}}
\]

As \( n \to \infty \), the terms \( \frac{4}{n^2} \) and \( \frac{1}{n^2} \) approach 0, leaving us with:

\[
\lim_{n \to \infty} a_n = 3n + 2
\]

Thus, the sequence grows without bound, and it does not converge to a finite limit.

\textbf{Answer: No}

\subsection*{4. Sequence \( a_n = \log_{n+1}(3) \)}

Finally, we analyze the sequence \( a_n = \log_{n+1}(3) \). Using the change of base formula for logarithms:

\[
a_n = \log_{n+1}(3) = \frac{\ln(3)}{\ln(n+1)}
\]

As \( n \to \infty \), \( \ln(n+1) \) grows indefinitely, so:

\[
\lim_{n \to \infty} \frac{\ln(3)}{\ln(n+1)} = 0
\]

Thus, the sequence \( a_n = \log_{n+1}(3) \) converges to 0.

\textbf{Answer: Yes}

\end{document}
